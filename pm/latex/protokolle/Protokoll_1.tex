%!TEX root = ../main.tex

\newcommand{\Titel}{1. Protokoll}
\newcommand{\Teilnehmer}{Jonas Bürgel, Patrick Welter, Markus Reischl}
\newcommand{\DatumUndZeit}{15.09.2021 14:00-14:30}
\newcommand{\Ort}{Zoom Meeting}
\newcommand{\Thema}{Projekt aufsetzen und Thema einreichen}
%!TEX root = ../main.tex

\begin{titlepage}
	\begin{center}
		\vspace*{-2.5cm}
		\hfill	\includegraphics[width=5cm]{images/dhbw.png}\\[5cm]
		
		{\Huge\scshape \Titel}\\[1cm]
		{\large für die Erstellung der Studienarbeit}\\[0.5cm]
		\vspace{1cm}
		
		\vfill
	\end{center}

	\begin{tabular}{l@{\hspace{2cm}}l}
		\bfseries Datum    & 17. Oktober 2021 \\
		\bfseries Betreuer & \Betreuer        \\
		\bfseries Kurs     & \Kursbezeichnung \\
		\bfseries Autoren  & \Autoren
	\end{tabular}

\end{titlepage}


\section{Anforderungsanalyse}
\begin{description}
	\item[Geometrische Formen] \hfill \\
	Auf den Trainings- und generierten Daten sollen geometrische Formen abgebildet sein. Die Formen unterscheiden sich in Art, Größe und Position.
	
	\item[Labels für Bilder] \hfill \\
	Die Bilder sollen gelabelt sein. Durch die Labels ist es später möglich, dem GAN mitzuteilen, es soll zum Beispiel ein Rechteck oder ein Kreis erzeugen.
\end{description}

\section{Organisatorisches}
\begin{description}
	\item[Projektmanagemente] \hfill \\
	Nutzung von einem SCRUM ähnlichen Prinzip mit wöchentlichen Sprints. Es existiert ein Issue-Board in GitHub, das die Aufgaben für jede Woche enthält.
	
	\item[Versionsverwaltung] \hfill \\
	Verwendung von GitHub. Markus Reischl hat kein Interesse an Repository, fertige PDFs werden per Email an ihn gesendet.
	
	\item[Treffen] \hfill \\
	Treffen mit Betreuer finden auf Nachfrage statt. Es gibt einen regelmäßigen Arbeitstermin jeden Donnerstag, der auch für Sprintplanung genutzt wird.
\end{description}

\section{Beschlüsse und Anmerkungen}
\begin{description}
	\item[Zeitplan] \hfill \\
	Bis Ende November soll bereits ein funktionierender Prototyp existieren. Zudem soll bereits ein großer Teil der Studienarbeit verfasst sein.
	\item[Synthetische Trainingsbilder] \hfill \\
	Die Trainingsbilder sollen synthetisch sein. Auf den gelabelten Bildern sollen zufällige geometrische Figuren abgebildet sein.
\end{description}
%Meilensteine
%Besondere Erkenntnisse der letzten Woche
%Beschlüsse, Änderungen, Vorgaben
%Kritische Probleme 

\section{Aufgabenverteilung und weiteres Vorgehen}
\begin{description}
	\item[Thema] \hfill \todoperson{Jonas, Patrick}\\ 
	Thema ausformulieren und einreichen. Bevor die Einreichung erfolgt, noch einmal per Email Rücksprache mit Markus Reischl halten.
	
	\item[Latex] \hfill \todoperson{Jonas, Patrick}\\
	Markus Reischl hat eine Vorlage mit Überschrift gesendet. Diese sollen in ein Latex-Dokument übernommen werden. Außerdem muss eine Struktur für die Studienarbeit geschaffen werden, die es erlaubt, simultan zu schreiben. 
	
	\item[Tensorflow] \hfill \todoperson{Jonas, Patrick}\\
	Für GPU-Betrieb installieren.
	
	\item[GAN + Synthetische Bilder] \hfill \todoperson{Jonas, Patrick}\\
	Erste Prototypen erstellen. Dafür sollen gelabelte und deterministisch zufällige Bilder in einen Ordner generiert werden. Die Bilder sollen zufällige geometrische Formen enthalten, die sich an unterschiedlichen Positionen befinden sollen.
	
	\item[Protokoll] \hfill \todoperson{Jonas}\\ 
	Aktuelles Protokoll verfassen und abschicken.
	
	\item[Nächstes Treffen] Sprint Planning \& Review am Donnerstag, 21.10.2021 um 10 Uhr 
\end{description}


