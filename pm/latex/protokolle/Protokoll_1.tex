%!TEX root = ../main.tex

\newcommand{\Titel}{1. Protokoll}
\newcommand{\Teilnehmer}{Jonas Bürgel, Patrick Welter}
\newcommand{\DatumUndZeit}{18.10.2021 20:00-20:15}
\newcommand{\Ort}{Discord Meeting}
\newcommand{\Thema}{Literaturrecherche}
%!TEX root = ../main.tex

\begin{titlepage}
	\begin{center}
		\vspace*{-2.5cm}
		\hfill	\includegraphics[width=5cm]{images/dhbw.png}\\[5cm]
		
		{\Huge\scshape \Titel}\\[1cm]
		{\large für die Erstellung der Studienarbeit}\\[0.5cm]
		\vspace{1cm}
		
		\vfill
	\end{center}

	\begin{tabular}{l@{\hspace{2cm}}l}
		\bfseries Datum    & 17. Oktober 2021 \\
		\bfseries Betreuer & \Betreuer        \\
		\bfseries Kurs     & \Kursbezeichnung \\
		\bfseries Autoren  & \Autoren
	\end{tabular}

\end{titlepage}


\section{Aufgaben der letzten Besprechung}
\begin{description}
	\item[Thema] Thema ausformulieren und einreichen. \todoperson{Jonas, Patrick} \fullcheck
	\item[Latex] Latex-Dokument nach Vorlage aufsetzen. \todoperson{Jonas, Patrick} \fullcheck
	\item[Tensorflow] Für GPU-Betrieb installieren. \todoperson{Jonas, Patrick} \fullcheck
	\item[Protokoll] \todoperson{Jonas} \fullcheck
\end{description}

\section{Beschlüsse und Anmerkungen}
\begin{description}[style=nextline]
	\item[Recherche]
	Es ist nur sehr wenig Wissen im Bereich Neuronale Netze oder GANs im Team vorhanden.
	Deshalb wird sich zunächst auf eine Grundlagenrecherche konzentriert.
\end{description}
%Meilensteine
%Besondere Erkenntnisse der letzten Woche
%Beschlüsse, Änderungen, Vorgaben
%Kritische Probleme 

\section{Qualitätssichernde Maßnahmen}
Besprechung möglicher auftretender Risiken und Absicherung der Qualitätssichernden Maßnahmen
\begin{description}[style=nextline]
	\item[Review und Dokumentation \hfill \fullcheck]
	Der nächste Sprint wurde gemäß des Zeitplans geplant und alle Beschlüsse dokumentiert. Der aktuelle Zeitplan kann eingehalten werden.
	%Überprüfung und ggf. Anpassung des Zeitplans und der Meilensteine. Dokumentation des aktuellen Stands und detaillierte Planung des nächsten Sprints.
	\item[Risikoanalyse \hfill \fullcheck]
	Während der Recherche ist das Risiko \textit{Komplexität} aufgetreten.
	Es stellt derzeit kein kritisches Problem dar.
	Durch zusätzlich zukünftige Recherchen kann es minimiert werden.
	%wöchentliche Analyse der definierten Risiken. Frühzeitiges Erkennen und Bekämpfen aufkommender Probleme.
	\item[Tests/Kontrollen \hfill \fullcheck]
	Tests sind bisher noch nicht notwendig.
	%Durchführung von Unit-Tests und Integration-Tests. Implementierung und Nutzung einer Testumgebung (Nutzeroberfläche) zur Darstellung von Ergebnissen.
	
\end{description}

\section{Aufgabenverteilung und weiteres Vorgehen}
\begin{description}[style=nextline]
	\item[Testdatengenerator \todoperson{Jonas, Patrick}] 
	Weitere Informationen über GANs sammeln.
	
	\item[Projekt Aufsetzen \todoperson{Jonas, Patrick}] 
	Zwar ist Tensorflow schon installiert, aber es muss noch ein Workspace für den Code und die Latex-Dokumente erstellt werden.
	
	\item[Protokoll \todoperson{Jonas}]
	Aktuelles Protokoll verfassen und abschicken.
\end{description}
\begin{description}
	\item[Nächstes Treffen] Sprint Planning \& Review am 01.11.2021
\end{description}

