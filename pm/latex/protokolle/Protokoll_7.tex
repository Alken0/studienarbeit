%!TEX root = ../main.tex

\newcommand{\Titel}{7. Protokoll}
\newcommand{\Teilnehmer}{Jonas Bürgel, Markus Reischl, Patrick Welter}
\newcommand{\DatumUndZeit}{07.03.2022 20:00-20:15}
\newcommand{\Ort}{Zoom Meeting}
\newcommand{\Thema}{Erweiterung GAN: Anpassungen zur Qualitätssteigerung}
%!TEX root = ../main.tex

\begin{titlepage}
	\begin{center}
		\vspace*{-2.5cm}
		\hfill	\includegraphics[width=5cm]{images/dhbw.png}\\[5cm]
		
		{\Huge\scshape \Titel}\\[1cm]
		{\large für die Erstellung der Studienarbeit}\\[0.5cm]
		\vspace{1cm}
		
		\vfill
	\end{center}

	\begin{tabular}{l@{\hspace{2cm}}l}
		\bfseries Datum    & 17. Oktober 2021 \\
		\bfseries Betreuer & \Betreuer        \\
		\bfseries Kurs     & \Kursbezeichnung \\
		\bfseries Autoren  & \Autoren
	\end{tabular}

\end{titlepage}


\section{Aufgaben der letzten Besprechung}
\begin{description}
	\item[Zwischenstand] \todoperson{Jonas} \fullcheck
	\item[Erweiterung des Testdatengenerators] gelabelte Daten \todoperson{Jonas, Patrick} \fullcheck
	\item[Dokumentation] Stand der Technik \todoperson{Jonas, Patrick} \fullcheck
	\item[Protokoll] \todoperson{Jonas} \fullcheck
\end{description}

\section{Meilensteine}
\begin{description}[style=nextline]
	\item[Komplexer Trainingsworkflow mit gelabelten Daten \hfill \fullcheck]
	Die Erstellung von gelabelten Testdaten und das label-spezifische Training ist nun möglich.
	Dadurch existiert ein komplexer Workflow.
\end{description}
%Meilensteine
%Besondere Erkenntnisse der letzten Woche
%Beschlüsse, Änderungen, Vorgaben
%Kritische Probleme 

\section{Qualitätssichernde Maßnahmen}
Besprechung möglicher auftretender Risiken und Absicherung der Qualitätssichernden Maßnahmen
\begin{description}[style=nextline]
	\item[Review und Dokumentation \hfill \fullcheck]
	Die Dokumentation erfolgt, das Kapitel \textit{Stand der Technik} ist abgeschlossen.
	
	%Überprüfung und ggf. Anpassung des Zeitplans und der Meilensteine. Dokumentation des aktuellen Stands und detaillierte Planung des nächsten Sprints.
	\item[Risikoanalyse \hfill \fullcheck]
	Keins der beschriebenen Risiken stellt zu diesem Zeitpunkt ein Problem dar.
	
	\item[Pair-Programming \hfill \fullcheck]
	Mithilfe von Pair-Programming wurde der Code durch das 4-Augen-Prinzip überprüft.
	
	%wöchentliche Analyse der definierten Risiken. Frühzeitiges Erkennen und Bekämpfen aufkommender Probleme.
	\item[Tests/Kontrollen \hfill \fullcheck]
	Die manuelle Auswertung der generierten Bilder des GANs liefert kein zufriedenstellendes Ergebnis.
	Jedoch funktioniert der Trainingsablauf und somit ist das Ziel erreicht.
	Die Qualität der Bilder kann nach und nach in den nächsten Wochen verbessert werden.
	Es handelt sich um kein schwerwiegendes oder kritisches Problem.
	%Durchführung von Unit-Tests und Integration-Tests. Implementierung und Nutzung einer Testumgebung (Nutzeroberfläche) zur Darstellung von Ergebnissen.
	
\end{description}

\section{Aufgabenverteilung und weiteres Vorgehen}
\begin{description}[style=nextline]
	\item[Erweiterung des GANs zu CGAN \todoperson{Jonas, Patrick}] 
	Die Labels werden zurzeit nur in den Testdaten mitgeneriert.
	Das GAN muss so weit angepasst werden, dass es die Labels mit berücksichtigt.
	
	\item[Dokumentation \todoperson{Jonas, Patrick}]
	Übertragung der bisher gesammelten Erkenntnisse in das Latex-Dokument.
	
	\item[Protokoll \todoperson{Jonas}]
	Aktuelles Protokoll verfassen und abschicken.
	Dabei handelt es sich insbesondere um die Kapitel \textit{Methodik} und \textit{Implementierung}.
\end{description}
\begin{description}
	\item[Nächstes Treffen] Sprint Planning \& Review am 21.02.2021
\end{description}

