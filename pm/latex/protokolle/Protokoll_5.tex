%!TEX root = ../main.tex

\newcommand{\Titel}{5. Protokoll}
\newcommand{\Teilnehmer}{Jonas Bürgel, Patrick Welter}
\newcommand{\DatumUndZeit}{13.12.2021 20:00-20:15}
\newcommand{\Ort}{Discord Meeting}
\newcommand{\Thema}{Erweiterung GAN: gelabelte Daten}
%!TEX root = ../main.tex

\begin{titlepage}
	\begin{center}
		\vspace*{-2.5cm}
		\hfill	\includegraphics[width=5cm]{images/dhbw.png}\\[5cm]
		
		{\Huge\scshape \Titel}\\[1cm]
		{\large für die Erstellung der Studienarbeit}\\[0.5cm]
		\vspace{1cm}
		
		\vfill
	\end{center}

	\begin{tabular}{l@{\hspace{2cm}}l}
		\bfseries Datum    & 17. Oktober 2021 \\
		\bfseries Betreuer & \Betreuer        \\
		\bfseries Kurs     & \Kursbezeichnung \\
		\bfseries Autoren  & \Autoren
	\end{tabular}

\end{titlepage}


\section{Aufgaben der letzten Besprechung}
\begin{description}
	\item[Erweiterung des Testdatengenerators] gelabelte Daten \todoperson{Jonas, Patrick} \fullcheck
	\item[Dokumentation] Methodik, Implementierung \todoperson{Jonas, Patrick} \halfcheck
	\item[Protokoll] \todoperson{Jonas} \fullcheck
\end{description}

\section{Meilensteine}
\begin{description}[style=nextline]
	\item[Gelabelte Testdaten \hfill \fullcheck]
	Die Testdaten können mit unterschiedlichen Formen generiert werden.
	Es existieren die Formen Quadrat, Dreieck und Kreis, die jeweils unterschiedlich groß und an unterschiedlichen Postionen im Bild abgebildet sein können.
	Die Bilder sind durchnummeriert, das Label ergibt sich aus dem Ordner, in dem das Bild liegt (circle für Kreis,...).
\end{description}
%Meilensteine
%Besondere Erkenntnisse der letzten Woche
%Beschlüsse, Änderungen, Vorgaben
%Kritische Probleme 

\section{Qualitätssichernde Maßnahmen}
Besprechung möglicher auftretender Risiken und Absicherung der Qualitätssichernden Maßnahmen
\begin{description}[style=nextline]
	\item[Review und Dokumentation \hfill \fullcheck]
	Die Dokumentation erfolgt, die Kapitel sind allerdings noch nicht abgeschlossen.
	Es konnte verhältnismäßig viel geschrieben werden, da die Implementierung nicht so aufwändig war, wie gedacht.
	
	%Überprüfung und ggf. Anpassung des Zeitplans und der Meilensteine. Dokumentation des aktuellen Stands und detaillierte Planung des nächsten Sprints.
	\item[Risikoanalyse \hfill \fullcheck]
	Keins der beschriebenen Risiken stellt zu diesem Zeitpunkt ein Problem dar.
	
	\item[Pair-Programming \hfill \fullcheck]
	Mithilfe von Pair-Programming wurde der Code durch das 4-Augen-Prinzip überprüft.
	
	%wöchentliche Analyse der definierten Risiken. Frühzeitiges Erkennen und Bekämpfen aufkommender Probleme.
	\item[Tests/Kontrollen \hfill \fullcheck]
	Tests für den Generator konnten geschrieben und ausgeführt werden.
	%Durchführung von Unit-Tests und Integration-Tests. Implementierung und Nutzung einer Testumgebung (Nutzeroberfläche) zur Darstellung von Ergebnissen.
	
\end{description}

\section{Aufgabenverteilung und weiteres Vorgehen}
\begin{description}[style=nextline]
	\item[Literaturrecherche zu CGAN \todoperson{Jonas, Patrick}]
	Es müssen mehr Informationen zu CGANs gesammelt werden, um diese korrekt zu implementieren.
	
	\item[Erweiterung des GANs zu CGAN \todoperson{Jonas, Patrick}] 
	Die Labels werden zurzeit nur in den Testdaten mitgeneriert.
	Das GAN muss so weit angepasst werden, dass es die Labels mit berücksichtigt.
	
	\item[Dokumentation \todoperson{Jonas, Patrick}]
	Übertragung der bisher gesammelten Erkenntnisse in das Latex-Dokument.
	Dabei handelt es sich insbesondere um die Kapitel \textit{Stand der Technik}, \textit{Methodik} und \textit{Implementierung}.
	
	\item[Protokoll \todoperson{Jonas}]
	Aktuelles Protokoll verfassen und abschicken.
\end{description}
\begin{description}
	\item[Nächstes Treffen] Sprint Planning \& Review am 21.02.2022
\end{description}

