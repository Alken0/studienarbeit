%!TEX root = ../main.tex

\newcommand{\Titel}{4. Protokoll}
\newcommand{\Teilnehmer}{Jonas Bürgel, Patrick Welter}
\newcommand{\DatumUndZeit}{29.11.2021 20:00-20:15}
\newcommand{\Ort}{Discord Meeting}
\newcommand{\Thema}{Entwicklung GAN}
%!TEX root = ../main.tex

\begin{titlepage}
	\begin{center}
		\vspace*{-2.5cm}
		\hfill	\includegraphics[width=5cm]{images/dhbw.png}\\[5cm]
		
		{\Huge\scshape \Titel}\\[1cm]
		{\large für die Erstellung der Studienarbeit}\\[0.5cm]
		\vspace{1cm}
		
		\vfill
	\end{center}

	\begin{tabular}{l@{\hspace{2cm}}l}
		\bfseries Datum    & 17. Oktober 2021 \\
		\bfseries Betreuer & \Betreuer        \\
		\bfseries Kurs     & \Kursbezeichnung \\
		\bfseries Autoren  & \Autoren
	\end{tabular}

\end{titlepage}


\section{Aufgaben der letzten Besprechung}
\begin{description}
	\item[Entwicklung GAN] \todoperson{Jonas, Patrick} \fullcheck
	\item[Dokumentation] S.d.T., Methodik, Implementierung \todoperson{Jonas, Patrick} \halfcheck
	\item[Protokoll] \todoperson{Jonas} \fullcheck
\end{description}

\section{Beschlüsse und Anmerkungen}
\begin{description}[style=nextline]
	\item[Pair-Programming]
	Pair-Programming hat sich auch bei der GAN-Implementierung als sehr vorteilhaft erwiesen.
	Viele Probleme konnten so schneller gelöst werden.
	
	\item[GAN-Tutorials]
	Es gibt sehr viele Tutorials zu GANs im Internet, die den Einstieg stark erleichtern.
	Diese können auch für andere Projekte verwendet werden und funktionieren verhältnismäßig gut.
\end{description}

\section{Meilensteine}
\begin{description}[style=nextline]
	\item[Einfacher Trainingsworkflow mit Testdaten und Generierung \hfill \fullcheck]
	Es konnten bereits seit dem Letzten Meilenstein Testbilder für das Training des GANs generiert werden.
	Nun ist es auch möglich, mithilfe der Bilder das GAN zu trainieren.
	
	Dadurch ist ein erster einfacher Ablauf vorhanden.
\end{description}
%Meilensteine
%Besondere Erkenntnisse der letzten Woche
%Beschlüsse, Änderungen, Vorgaben
%Kritische Probleme 

\section{Qualitätssichernde Maßnahmen}
Besprechung möglicher auftretender Risiken und Absicherung der Qualitätssichernden Maßnahmen
\begin{description}[style=nextline]
	\item[Review und Dokumentation \hfill \fullcheck]
	Die Dokumentation erfolgt, die Kapitel sind allerdings noch nicht abgeschlossen.
	
	%Überprüfung und ggf. Anpassung des Zeitplans und der Meilensteine. Dokumentation des aktuellen Stands und detaillierte Planung des nächsten Sprints.
	\item[Risikoanalyse \hfill \fullcheck]
	Keins der beschriebenen Risiken stellt zu diesem Zeitpunkt ein Problem dar.
	
	\item[Pair-Programming \hfill \fullcheck]
	Mithilfe von Pair-Programming wurde der Code durch das 4-Augen-Prinzip überprüft.
	
	%wöchentliche Analyse der definierten Risiken. Frühzeitiges Erkennen und Bekämpfen aufkommender Probleme.
	\item[Tests/Kontrollen \hfill \fullcheck]
	Es können zwar die generierten Bilder des GANs händisch kontrolliert werden, aber keine Tests geschrieben werden, da es keine klare Abtrennung zwischen 'gut' und 'schlecht' gibt.
	Deshalb wurden die Bilder manuell durchgegangen und ausgewertet.
	%Durchführung von Unit-Tests und Integration-Tests. Implementierung und Nutzung einer Testumgebung (Nutzeroberfläche) zur Darstellung von Ergebnissen.
	
\end{description}

\section{Aufgabenverteilung und weiteres Vorgehen}
\begin{description}[style=nextline]
	\item[Erweiterung des Testdatengenerators mit gelabelten Daten \todoperson{Jonas, Patrick}] 
	Generierung der Formen: Dreieck, Quadrat und Kreis in einer Ordnerstruktur, die das automatische Einlesen gelabelter Daten erlaubt.
	
	\item[Dokumentation \todoperson{Jonas, Patrick}]
	Übertragung der bisher gesammelten Erkenntnisse in das Latex-Dokument.
	Dabei handelt es sich insbesondere um die Kapitel \textit{Methodik} und \textit{Implementierung}.
	
	\item[Protokoll \todoperson{Jonas}]
	Aktuelles Protokoll verfassen und abschicken.
\end{description}
\begin{description}
	\item[Nächstes Treffen] Sprint Planning \& Review am 13.12.2021
\end{description}

