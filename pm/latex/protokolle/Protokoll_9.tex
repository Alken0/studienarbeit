%!TEX root = ../main.tex

\newcommand{\Titel}{9. Protokoll}
\newcommand{\Teilnehmer}{Jonas Bürgel, Markus Reischl, Patrick Welter}
\newcommand{\DatumUndZeit}{04.04.2022 20:00-20:15}
\newcommand{\Ort}{Zoom Meeting}
\newcommand{\Thema}{Hyperparameteroptimierung}
%!TEX root = ../main.tex

\begin{titlepage}
	\begin{center}
		\vspace*{-2.5cm}
		\hfill	\includegraphics[width=5cm]{images/dhbw.png}\\[5cm]
		
		{\Huge\scshape \Titel}\\[1cm]
		{\large für die Erstellung der Studienarbeit}\\[0.5cm]
		\vspace{1cm}
		
		\vfill
	\end{center}

	\begin{tabular}{l@{\hspace{2cm}}l}
		\bfseries Datum    & 17. Oktober 2021 \\
		\bfseries Betreuer & \Betreuer        \\
		\bfseries Kurs     & \Kursbezeichnung \\
		\bfseries Autoren  & \Autoren
	\end{tabular}

\end{titlepage}


\section{Aufgaben der letzten Besprechung}
\begin{description}
	\item[Hyperparameteroptimierung] aller Architekturen \todoperson{Patrick} \halfcheck
	\item[Besprechungstermin] mit Markus \todoperson{Jonas} \fullcheck
	\item[Dokumentation] Methodik und Implementierung \todoperson{Jonas, Patrick} \halfcheck
	\item[Protokoll] \todoperson{Jonas} \fullcheck
\end{description}

%Meilensteine
%Besondere Erkenntnisse der letzten Woche
%Beschlüsse, Änderungen, Vorgaben
%Kritische Probleme 

\section{Beschlüsse und Anmerkungen}
\begin{description}[style=nextline]
	\item[Markus: Ergebnisse]
	Es sind keine sehr guten Ergebnisse nötig, eine Beschreibung der Architekturen ist ausreichend.
\end{description}

\section{Qualitätssichernde Maßnahmen}
Besprechung möglicher auftretender Risiken und Absicherung der Qualitätssichernden Maßnahmen.
\begin{description}[style=nextline]
	\item[Review und Dokumentation \hfill \fullcheck]
	Die Dokumentation erfolgt, das Kapitel \textit{Stand der Technik} ist abgeschlossen.
	
	%Überprüfung und ggf. Anpassung des Zeitplans und der Meilensteine. Dokumentation des aktuellen Stands und detaillierte Planung des nächsten Sprints.
	\item[Risikoanalyse \hfill \fullcheck]
	Obwohl die Bilder trotz Hyperparametersuche nicht optimal aussehen, reichen sie für die Studienarbeit.
	Damit ist das Risiko \textit{Komplexität} kein Problem mehr.
	
	Andere Risikofaktoren sind derzeit nicht problematisch einzustufen.
	
	\item[Pair-Programming \hfill \fullcheck]
	Mithilfe von Pair-Programming wurde der Code durch das 4-Augen-Prinzip überprüft.
	
	%wöchentliche Analyse der definierten Risiken. Frühzeitiges Erkennen und Bekämpfen aufkommender Probleme.
	%Durchführung von Unit-Tests und Integration-Tests. Implementierung und Nutzung einer Testumgebung (Nutzeroberfläche) zur Darstellung von Ergebnissen.
	
\end{description}

\section{Aufgabenverteilung und weiteres Vorgehen}
\begin{description}[style=nextline]
	\item[Ergebnisse festhalten \todoperson{Jonas, Patrick}] 
	Die Ergebnisse der Hyperparametersuche in der Studienarbeit festhalten
	
	\item[Methodik: Hyperparametersuche \todoperson{Jonas}]
	Wir haben uns für das Verfahren Gridsearch entschieden, dies muss in der Studienarbeit festgehalten werden.
	
	\item[Dokumentation \todoperson{Jonas, Patrick}]
	Übertragung der bisher gesammelten Erkenntnisse in das Latex-Dokument.
	
	\item[Protokoll \todoperson{Jonas}]
	Aktuelles Protokoll verfassen und abschicken.
\end{description}
\begin{description}
	\item[Nächstes Treffen] Sprint Planning \& Review am 18.04.2022
\end{description}

