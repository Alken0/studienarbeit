%!TEX root = ../main.tex

\newcommand{\Titel}{3. Protokoll}
\newcommand{\Teilnehmer}{Jonas Bürgel, Patrick Welter, Markus Reischl}
\newcommand{\DatumUndZeit}{21.10.2021 16:00-16:15}
\newcommand{\Ort}{Zoom Meeting}
\newcommand{\Thema}{Fertigstellung Prototyp und Beginn Schreibarbeit}
%!TEX root = ../main.tex

\begin{titlepage}
	\begin{center}
		\vspace*{-2.5cm}
		\hfill	\includegraphics[width=5cm]{images/dhbw.png}\\[5cm]
		
		{\Huge\scshape \Titel}\\[1cm]
		{\large für die Erstellung der Studienarbeit}\\[0.5cm]
		\vspace{1cm}
		
		\vfill
	\end{center}

	\begin{tabular}{l@{\hspace{2cm}}l}
		\bfseries Datum    & 17. Oktober 2021 \\
		\bfseries Betreuer & \Betreuer        \\
		\bfseries Kurs     & \Kursbezeichnung \\
		\bfseries Autoren  & \Autoren
	\end{tabular}

\end{titlepage}


\section{Aufgaben der letzten Besprechung}
\begin{description}
	\item[X] \hfill \todoperson{Jonas, Patrick} \fullcheck
	\item[Y] \hfill \todoperson{Jonas, Patrick} \halfcheck
	\item[Z] \hfill \todoperson{Jonas, Patrick} \nocheck
	\item[Protokoll] \hfill \todoperson{Jonas} \fullcheck
\end{description}

\section{Beschlüsse und Anmerkungen}
%Meilensteine
%Besondere Erkenntnisse der letzten Woche
%Beschlüsse, Änderungen, Vorgaben
%Kritische Probleme 

\section{Qualitätssichernde Maßnahmen}
Besprechung möglicher auftretender Risiken und Absicherung der Qualitätssichernden Maßnahmen
\begin{description}
	\item[Review und Dokumentation] \hfill \fullcheck\\
	Der nächste Sprint wurde gemäß des Zeitplans geplant und alle Beschlüsse dokumentiert. Der aktuelle Zeitplan kann eingehalten werden.
	%Überprüfung und ggf. Anpassung des Zeitplans und der Meilensteine. Dokumentation des aktuellen Stands und detaillierte Planung des nächsten Sprints.
	\item[Risikoanalyse] \hfill \fullcheck \\ 
	Keines der Risiken stellt momentan ein Problem dar.
	%wöchentliche Analyse der definierten Risiken. Frühzeitiges Erkennen und Bekämpfen aufkommender Probleme.
	\item[Teamarbeit] \hfill \fullcheck \\ 
	Beide Teammitglieder befinden sich auf demselben Wissensstand. Pair Programming wurde erfolgreich durchgeführt.
	%Pair Programming wo dies sinnvoll ist, Teilen von Erkenntnissen und Abgleichung des Wissenstands. Korrekturlesen der schriftlichen Ausarbeitung des jeweils anderen.
	\item[Testen] \hfill \fullcheck \\ 
	Tests sind bisher noch nicht notwendig.
	%Durchführung von Unit-Tests und Integration-Tests. Implementierung und Nutzung einer Testumgebung (Nutzeroberfläche) zur Darstellung von Ergebnissen.
	\item[Rücksprachen mit dem Auftraggeber] \hfill \fullcheck \\ 
	Das nächste Treffen mit dem Betreuer wird geplant, sobald ein erstes Konzept für [xyz] entworfen und implementiert wurde.
	%Regelmäßige Besprechungen mit dem Betreuer, um den aktuellen Stand mit den Erwartungen abzugleichen und Missverständnissen vorzubeugen.
	
\end{description}

\section{Aufgabenverteilung und weiteres Vorgehen}
\begin{description}
	\item[X] \hfill \todoperson{Jonas, Patrick}\\
	
	\item[Protokoll] \hfill \todoperson{Jonas}\\Aktuelles Protokoll verfassen und abschicken 
	\item[Nächstes Treffen] Sprint Planning \& Review am Sonntag, 24.10.2021 um 9 Uhr 
\end{description}

