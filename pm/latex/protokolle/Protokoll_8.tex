%!TEX root = ../main.tex

\newcommand{\Titel}{8. Protokoll}
\newcommand{\Teilnehmer}{Jonas Bürgel, Patrick Welter}
\newcommand{\DatumUndZeit}{21.03.2022 20:00-20:15}
\newcommand{\Ort}{Discord Meeting}
\newcommand{\Thema}{Erweiterung GAN: Anpassungen zur Qualitätssteigerung}
%!TEX root = ../main.tex

\begin{titlepage}
	\begin{center}
		\vspace*{-2.5cm}
		\hfill	\includegraphics[width=5cm]{images/dhbw.png}\\[5cm]
		
		{\Huge\scshape \Titel}\\[1cm]
		{\large für die Erstellung der Studienarbeit}\\[0.5cm]
		\vspace{1cm}
		
		\vfill
	\end{center}

	\begin{tabular}{l@{\hspace{2cm}}l}
		\bfseries Datum    & 17. Oktober 2021 \\
		\bfseries Betreuer & \Betreuer        \\
		\bfseries Kurs     & \Kursbezeichnung \\
		\bfseries Autoren  & \Autoren
	\end{tabular}

\end{titlepage}


\section{Aufgaben der letzten Besprechung}
\begin{description}
	\item[Verbesserung der Architektur] 4 Architekturen getestet \todoperson{Jonas, Patrick} \halfcheck
	\item[Verbesserung der Anmerkungen] Markus Kritik \todoperson{Jonas, Patrick} \fullcheck
	\item[Dokumentation] Methodik und Implementierung \todoperson{Jonas, Patrick} \halfcheck
	\item[Protokoll] \todoperson{Jonas} \fullcheck
\end{description}

%Meilensteine
%Besondere Erkenntnisse der letzten Woche
%Beschlüsse, Änderungen, Vorgaben
%Kritische Probleme 

\section{Beschlüsse und Anmerkungen}
\begin{description}[style=nextline]
	\item[Architekturen]
	Damit der Trainingsablauf nicht jedes mal geändert werden muss, sollen alle Architekturen die gleichen Ein- und Ausgaben erhalten. 
	Das muss auch in der Dokumentation festgehalten werden!
	
	\item[Trainingsergebnisse festhalten]
	Da die jetzigen Ergebnisse auch für die Analyse der Studienarbeit relevant sind, muss immer ein Commit mit dem derzeitigen Code-Stand + die Ergebnisse in Git hochgeladen werden. 
	So können die Ergebnisse später besser nachvollzogen werden.
\end{description}

\section{Qualitätssichernde Maßnahmen}
Besprechung möglicher auftretender Risiken und Absicherung der Qualitätssichernden Maßnahmen.
\begin{description}[style=nextline]
	\item[Review und Dokumentation \hfill \fullcheck]
	Die Dokumentation erfolgt weiterhin.
	Es wird an den genannten Kapiteln weitergeschrieben und Korrektur gelesen.
	Der Umfang des Textes ist verhältnismäßig für den Stand der Arbeit, es besteht kein relevantes Risiko, dass die Arbeit nicht rechtzeitig fertig wird.
	
	%Überprüfung und ggf. Anpassung des Zeitplans und der Meilensteine. Dokumentation des aktuellen Stands und detaillierte Planung des nächsten Sprints.
	\item[Risikoanalyse \hfill \fullcheck]
	Die generierten Bilder sehen noch nicht zufriedenstellend aus, obwohl 4 stark unterschiedliche Architekturen ausprobiert wurden.
	Eventuell werden keine sehr guten Architekturen gefunden, was zum Risiko \textit{Komplexität} gehört.
	
	Als Gegenmaßnahme wird ein Gespräch mit Markus vereinbart werden, um über das weitere Vorgehen zu diskutieren.
	
	\item[Pair-Programming \hfill \fullcheck]
	Mithilfe von Pair-Programming wurde der Code durch das 4-Augen-Prinzip überprüft.
	
	%wöchentliche Analyse der definierten Risiken. Frühzeitiges Erkennen und Bekämpfen aufkommender Probleme.
	%Durchführung von Unit-Tests und Integration-Tests. Implementierung und Nutzung einer Testumgebung (Nutzeroberfläche) zur Darstellung von Ergebnissen.
	
\end{description}

\section{Aufgabenverteilung und weiteres Vorgehen}
\begin{description}[style=nextline]
	\item[Auswahl Verfahren Hyperparameteroptimierung \todoperson{Jonas, Patrick}]
	Es existieren verschiedene Verfahren zu Optimierung von Hyperparametern.
	Aus allen Verfahren muss ein passendes für die Studienarbeit ausgewählt und festgehalten werden.
	
	\item[Hyperparameteroptimierung aller Architekturen \todoperson{Patrick}] 
	Durch eine ausführliche Hyperparameter-Optimierung könnten die Architekturen bessere Ergebnisse liefern.
	Da keine Architektur bis jetzt sehr gute Ergebnisse gezeigt hat, werden alle Architekturen ausprobiert.
	
	\item[Besprechung mit Markus bezüglich Ergebnissen vereinbaren \todoperson{Jonas}]
	Bezüglich des Weiteren Vorgehens, falls keine guten Ergebnisse erreicht werden.
	
	\item[Dokumentation \todoperson{Jonas, Patrick}]
	Übertragung der bisher gesammelten Erkenntnisse in das Latex-Dokument.
	
	\item[Protokoll \todoperson{Jonas}]
	Aktuelles Protokoll verfassen und abschicken.
\end{description}
\begin{description}
	\item[Nächstes Treffen] Sprint Planning \& Review am 04.04.2022
\end{description}

