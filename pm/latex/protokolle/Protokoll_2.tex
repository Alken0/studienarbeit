%!TEX root = ../main.tex

\newcommand{\Titel}{2. Protokoll}
\newcommand{\Teilnehmer}{Jonas Bürgel, Patrick Welter, Markus Reischl}
\newcommand{\DatumUndZeit}{21.10.2021 16:00-16:15}
\newcommand{\Ort}{Zoom Meeting}
\newcommand{\Thema}{Fertigstellung Prototyp und Verfassen erster Seiten zur Studienarbeit}
%!TEX root = ../main.tex

\begin{titlepage}
	\begin{center}
		\vspace*{-2.5cm}
		\hfill	\includegraphics[width=5cm]{images/dhbw.png}\\[5cm]
		
		{\Huge\scshape \Titel}\\[1cm]
		{\large für die Erstellung der Studienarbeit}\\[0.5cm]
		\vspace{1cm}
		
		\vfill
	\end{center}

	\begin{tabular}{l@{\hspace{2cm}}l}
		\bfseries Datum    & 17. Oktober 2021 \\
		\bfseries Betreuer & \Betreuer        \\
		\bfseries Kurs     & \Kursbezeichnung \\
		\bfseries Autoren  & \Autoren
	\end{tabular}

\end{titlepage}


\section{Aufgaben der letzten Besprechung}
\begin{description}
	\item[Thema] Thema ausformulieren und einreichen. \hfill \todoperson{Jonas, Patrick} \fullcheck
	\item[Latex] Latex-Dokument nach Vorlage aufsetzen. \hfill \todoperson{Jonas, Patrick} \fullcheck
	\item[Tensorflow] Für GPU-Betrieb installieren. \hfill \todoperson{Jonas, Patrick} \fullcheck
	\item[GAN + Synthetische Bilder] Erste Prototypen erstellen. \hfill \todoperson{Jonas, Patrick} \halfcheck
	\item[Protokoll] \hfill \todoperson{Jonas} \fullcheck
\end{description}

\section{Beschlüsse und Anmerkungen}
\begin{description}
	\item[Anmerkung: GAN + Synthetische Bilder] \hfill \\
	Es konnten sowohl ein GAN als auch ein Generator für Synthetische Bilder geschrieben werden. Allerdings waren Code-Beispiele für das GAN komplizierter als gedacht, weswegen das GAN noch nicht mit den synthetischen Bildern lernt.
	
	\item[Pair-Programming] \hfill \\
	Es wurde festgestellt, dass die Arbeit relativ wenig reine Programmierarbeit beinhaltet. Stattdessen kommt es viel mehr auf die Konfiguration des GANs an. Die Programmierarbeiten werden alle in Pair-Programming stattfinden, damit ein gemeinsames einheitliches Verständnis für die Arbeit beibehalten wird.
\end{description}
%Meilensteine
%Besondere Erkenntnisse der letzten Woche
%Beschlüsse, Änderungen, Vorgaben
%Kritische Probleme 

\section{Qualitätssichernde Maßnahmen}
Besprechung möglicher auftretender Risiken und Absicherung der Qualitätssichernden Maßnahmen
\begin{description}
	\item[Review und Dokumentation] \hfill \fullcheck\\
	Der nächste Sprint wurde gemäß des Zeitplans geplant und alle Beschlüsse dokumentiert. Der aktuelle Zeitplan kann eingehalten werden.
	%Überprüfung und ggf. Anpassung des Zeitplans und der Meilensteine. Dokumentation des aktuellen Stands und detaillierte Planung des nächsten Sprints.
	\item[Risikoanalyse] \hfill \fullcheck \\ 
	Keines der Risiken stellt momentan ein Problem dar.
	%wöchentliche Analyse der definierten Risiken. Frühzeitiges Erkennen und Bekämpfen aufkommender Probleme.
	\item[Testen] \hfill \fullcheck \\ 
	Tests sind bisher noch nicht notwendig.
	%Durchführung von Unit-Tests und Integration-Tests. Implementierung und Nutzung einer Testumgebung (Nutzeroberfläche) zur Darstellung von Ergebnissen.
	\item[Rücksprachen mit dem Auftraggeber] \hfill \fullcheck \\ 
	Das nächste Treffen mit dem Betreuer wird geplant, sobald die folgenden Aufgaben zufriedenstellend erledigt sind. Ziel ist eine Fertigstellung zum folgenden Donnerstag.
	%Regelmäßige Besprechungen mit dem Betreuer, um den aktuellen Stand mit den Erwartungen abzugleichen und Missverständnissen vorzubeugen.
	
\end{description}

\section{Aufgabenverteilung und weiteres Vorgehen}
\begin{description}
	\item[Latex] 10 Seiten verfassen inklusive den vorgegebenen Überschriften.\hfill \todoperson{Jonas, Patrick}\\
	\item[GAN] GAN mit generierten Bildern verknüpfen und so einen ersten vollständigen Workflow schaffen. \hfill \todoperson{Jonas, Patrick}\\
	\item[Protokoll] \hfill \todoperson{Jonas}\\Aktuelles Protokoll verfassen und abschicken 
	\item[Nächstes Treffen] Sprint Planning \& Review am Donnerstag, 28.10.2021 um 10 Uhr 
\end{description}

