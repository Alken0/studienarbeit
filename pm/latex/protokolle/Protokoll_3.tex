%!TEX root = ../main.tex

\newcommand{\Titel}{3. Protokoll}
\newcommand{\Teilnehmer}{Jonas Bürgel, Patrick Welter}
\newcommand{\DatumUndZeit}{15.11.2021 20:00-20:15}
\newcommand{\Ort}{Discord Meeting}
\newcommand{\Thema}{Literaturrecherche}
%!TEX root = ../main.tex

\begin{titlepage}
	\begin{center}
		\vspace*{-2.5cm}
		\hfill	\includegraphics[width=5cm]{images/dhbw.png}\\[5cm]
		
		{\Huge\scshape \Titel}\\[1cm]
		{\large für die Erstellung der Studienarbeit}\\[0.5cm]
		\vspace{1cm}
		
		\vfill
	\end{center}

	\begin{tabular}{l@{\hspace{2cm}}l}
		\bfseries Datum    & 17. Oktober 2021 \\
		\bfseries Betreuer & \Betreuer        \\
		\bfseries Kurs     & \Kursbezeichnung \\
		\bfseries Autoren  & \Autoren
	\end{tabular}

\end{titlepage}


\section{Aufgaben der letzten Besprechung}
\begin{description}
	\item[Grundimplementierung Testdatengenerator] \todoperson{Jonas, Patrick} \fullcheck
	\item[Dokumentation] Stand der Technik \todoperson{Jonas, Patrick} \halfcheck
	\item[Protokoll] \todoperson{Jonas} \fullcheck
\end{description}

\section{Beschlüsse und Anmerkungen}
\begin{description}[style=nextline]
	\item[Pair-Programming]
	Für diese Arbeit muss nur sehr wenig programmiert werden.
	Deshalb macht es Sinn, wenn der Programmierteil gemeinsam erledigt wird.
	Dadurch ist auch eine zusätzliche Kontrolle durch das jeweilige Teammitglied gewährleistet.
\end{description}

\section{Meilensteine}
\begin{description}[style=nextline]
	\item[Aneignung von Grundlagen zu Neuronalen Netzen und GANs \hfill \halfcheck]
	Grundlagen konnten sich angeeignet werden, aber das Thema ist deutlich komplexer als erwartet.
	
	
	\item[Aufsetzen einer Entwicklungsumgebung \hfill \fullcheck]
	Es existiert eine funktionierende Entwicklungsumgebung bei allen Teammitgliedern.
\end{description}
%Meilensteine
%Besondere Erkenntnisse der letzten Woche
%Beschlüsse, Änderungen, Vorgaben
%Kritische Probleme 

\section{Qualitätssichernde Maßnahmen}
Besprechung möglicher auftretender Risiken und Absicherung der Qualitätssichernden Maßnahmen
\begin{description}[style=nextline]
	\item[Review und Dokumentation \hfill \fullcheck]
	Der nächste Sprint wurde gemäß des Zeitplans geplant und alle Beschlüsse dokumentiert.
	Es könnten in Zukunft Zeitprobleme aufgrund zusätzlicher Literaturrecherchen entstehen.
	Der zusätzliche Zeitaufwand sollte allerdings nicht kritisch sein, weshalb der aktuelle Zeitplan eingehalten werden kann.
	%Überprüfung und ggf. Anpassung des Zeitplans und der Meilensteine. Dokumentation des aktuellen Stands und detaillierte Planung des nächsten Sprints.
	\item[Risikoanalyse \hfill \fullcheck]
	Während der Recherche ist das Risiko \textit{Komplexität} aufgetreten.
	Es stellt derzeit kein kritisches Problem dar.
	Durch zusätzlich zukünftige Recherchen kann es minimiert werden.
	%wöchentliche Analyse der definierten Risiken. Frühzeitiges Erkennen und Bekämpfen aufkommender Probleme.
	\item[Tests/Kontrollen \hfill \fullcheck]
	Die Entwicklungsumgebung wurde durch das Ausführen eines Beispiels aus dem Internet getestet.
	Damit wurde die korrekte Pythoninstallation und Einrichtung von Tensorflow überprüft.
	%Durchführung von Unit-Tests und Integration-Tests. Implementierung und Nutzung einer Testumgebung (Nutzeroberfläche) zur Darstellung von Ergebnissen.
	
\end{description}

\section{Aufgabenverteilung und weiteres Vorgehen}
\begin{description}[style=nextline]
	\item[Implementierung Testdatengenerator \todoperson{Jonas, Patrick}] 
	Ein Generator zur Erzeugung von Bilder für das Training des GANs.
	
	\item[Dokumentation \todoperson{Jonas, Patrick}]
	Übertragung der bisher gesammelten Erkenntnisse in das Latex-Dokument.
	Dabei handelt es sich insbesondere um das Kapitel \textit{Stand der Technik}.
	
	\item[Protokoll \todoperson{Jonas}]
	Aktuelles Protokoll verfassen und abschicken.
\end{description}
\begin{description}
	\item[Nächstes Treffen] Sprint Planning \& Review am 29.11.2021
\end{description}

