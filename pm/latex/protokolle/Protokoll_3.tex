%!TEX root = ../main.tex

\newcommand{\Titel}{3. Protokoll}
\newcommand{\Teilnehmer}{Jonas Bürgel, Patrick Welter}
\newcommand{\DatumUndZeit}{15.11.2021 20:00-20:15}
\newcommand{\Ort}{Discord Meeting}
\newcommand{\Thema}{Entwicklung Testdatengenerator}
%!TEX root = ../main.tex

\begin{titlepage}
	\begin{center}
		\vspace*{-2.5cm}
		\hfill	\includegraphics[width=5cm]{images/dhbw.png}\\[5cm]
		
		{\Huge\scshape \Titel}\\[1cm]
		{\large für die Erstellung der Studienarbeit}\\[0.5cm]
		\vspace{1cm}
		
		\vfill
	\end{center}

	\begin{tabular}{l@{\hspace{2cm}}l}
		\bfseries Datum    & 17. Oktober 2021 \\
		\bfseries Betreuer & \Betreuer        \\
		\bfseries Kurs     & \Kursbezeichnung \\
		\bfseries Autoren  & \Autoren
	\end{tabular}

\end{titlepage}


\section{Aufgaben der letzten Besprechung}
\begin{description}
	\item[Grundimplementierung Testdatengenerator] \todoperson{Jonas, Patrick} \fullcheck
	\item[Dokumentation] Stand der Technik \todoperson{Jonas, Patrick} \halfcheck
	\item[Protokoll] \todoperson{Jonas} \fullcheck
\end{description}

\section{Beschlüsse und Anmerkungen}
\begin{description}[style=nextline]
	\item[Pair-Programming]
	Für diese Arbeit muss nur sehr wenig programmiert werden.
	Deshalb macht es Sinn, wenn der Programmierteil gemeinsam erledigt wird.
	
	Dadurch ist auch eine zusätzliche Kontrolle durch das jeweils andere Teammitglied gewährleistet.
	Das Schreiben von sehr schlechten Code kann so verhindert werden.
	
	\item[Schwierigkeiten mit Python]
	Die Programmiersprache Python ist keinem Teammitglied bekannt, aber Standard für die Implementierung von neuronalen Netzen.
	Das Erlernen ist etwas schwieriger als erwartet, aber vor allem durch Pair-Programming keine große Herausforderung.
\end{description}

\section{Meilensteine}
\begin{description}[style=nextline]
	\item[Grundimplementierung des Testdatengenerators \hfill \fullcheck]
	Es können Bilder generiert werden, die den Anforderungen für das Projekt entsprechen.
	Die Formen auf den Bildern sind teilweise konfigurierbar und teilweise zufallsgesteuert.
\end{description}
%Meilensteine
%Besondere Erkenntnisse der letzten Woche
%Beschlüsse, Änderungen, Vorgaben
%Kritische Probleme 

\section{Qualitätssichernde Maßnahmen}
Besprechung möglicher auftretender Risiken und Absicherung der Qualitätssichernden Maßnahmen
\begin{description}[style=nextline]
	\item[Review und Dokumentation \hfill \fullcheck]
	Der nächste Sprint wurde gemäß des Zeitplans geplant und alle Beschlüsse dokumentiert.
	Es könnten in Zukunft Zeitprobleme aufgrund von Python entstehen, das wäre aber sehr unwahrscheinlich.
	Das Wissen zu neuronalen Netzen ist bis jetzt ausreichend und es werden zunächst keine weiteren Recherchen benötigt.

	%Überprüfung und ggf. Anpassung des Zeitplans und der Meilensteine. Dokumentation des aktuellen Stands und detaillierte Planung des nächsten Sprints.
	\item[Risikoanalyse \hfill \fullcheck]
	Keins der beschriebenen Risiken stellt zu diesem Zeitpunkt ein Problem dar.
	
	%wöchentliche Analyse der definierten Risiken. Frühzeitiges Erkennen und Bekämpfen aufkommender Probleme.
	\item[Tests/Kontrollen \hfill \fullcheck]
	Es wurden Tests geschrieben und durch Pair-Programming der Code bereits beim Schreiben durch das 4-Augen Prinzip validiert.
	%Durchführung von Unit-Tests und Integration-Tests. Implementierung und Nutzung einer Testumgebung (Nutzeroberfläche) zur Darstellung von Ergebnissen.
	
\end{description}

\section{Aufgabenverteilung und weiteres Vorgehen}
\begin{description}[style=nextline]
	\item[Entwicklung GAN \todoperson{Jonas, Patrick}] 
	Eine erste Implementierung eines GANs.
	
	\item[Dokumentation \todoperson{Jonas, Patrick}]
	Übertragung der bisher gesammelten Erkenntnisse in das Latex-Dokument.
	Dabei handelt es sich insbesondere um die Kapitel \textit{Stand der Technik- Methodik}.
	
	\item[Protokoll \todoperson{Jonas}]
	Aktuelles Protokoll verfassen und abschicken.
\end{description}
\begin{description}
	\item[Nächstes Treffen] Sprint Planning \& Review am 29.11.2021
\end{description}

