%!TEX root = ../main.tex

\newcommand{\Titel}{Abschlussbericht}
%!TEX root = ../main.tex

\begin{titlepage}
	\begin{center}
		\vspace*{-2.5cm}
		\hfill	\includegraphics[width=5cm]{images/dhbw.png}\\[5cm]
		
		{\Huge\scshape \Titel}\\[1cm]
		{\large für die Erstellung der Studienarbeit}\\[0.5cm]
		\vspace{1cm}
		
		\vfill
	\end{center}

	\begin{tabular}{l@{\hspace{2cm}}l}
		\bfseries Datum    & 17. Oktober 2021 \\
		\bfseries Betreuer & \Betreuer        \\
		\bfseries Kurs     & \Kursbezeichnung \\
		\bfseries Autoren  & \Autoren
	\end{tabular}

\end{titlepage}


\section{Umsetzung gesetzter Ziele}
\subsection{Aufgabenstellung}
Ziel des Projekts war es, ein möglichst optimales GAN zur Erzeugung von Bildern zu entwickeln.
Die Bilder enthalten in beliebiger Größe und Position ein Dreieck, Quadrat oder Kreis.
Der Datensatz wird dafür extra generiert.
Beim Optimierungsprozess sollen auch die Hyperparameter für die Architekturen angepasst werden.
\subsection{Ergebnis}
Es existiert ein Datengenerator zur Erstellung der relevanten Bilder.
Außerdem wurden verschiedene Architekturen mit jeweils angepassten Hyperparametern trainiert und die Architekturen verglichen.
Bei den Architekturen handelt es sich um ein Dense-GAN und ein DC-GAN.
Die besten Ergebnisse liefert das Dense-GAN, das erkennbare Figuren erzeugen kann.
Die erzeugten Bilder sind jedoch qualitativ nicht so gut, wie die Trainingsdaten.
\subsection{Abgabebestätigung}
\begin{figure}[H]
	\centering
	\includegraphics[height=12cm]{./allgemein/Abgabebestätigung.png}
\end{figure}
\section{Reflexion der Vorgehensweise}
\subsection{Organisation/Zeitplan}
Der erstellte Zeitplan konnte überraschender Weise verhältnismäßig gut eingehalten werden, obwohl er über so einen langen Zeitraum ging.
Trotzdem war der eingebaute Puffer am Ende essentiell, bei wichtigeren Projekten sollte dieser eventuell noch größer ausfallen.
\subsection{Qualitätssichernde Maßnahmen}
Zunächst gab es große Sorge bezüglich der Komplexität der Aufgabe.
Die war tatsächlich auch ein Problem, aber nicht in dem Umfang, wie befürchtet.
Das lag hauptsächlich an mangelnden Kenntnissen im Bereich 'machine learning' und vor allem auch GANs im speziellen.
Das meiste Wissen zu den Themen wurde sich erst während der Arbeit angeeignet.
\newline

Die Hardware war so einschränkend wie von Anfang an einkalkuliert.
Mit deutlich bessere Hardware ist ein ausgiebigeres Training und eine ausführlichere Hyperparameteroptimierung möglich.
Dadurch entstehen dann auch bessere Ergebnisse.
\newline

Die anderen Risiken stellten alle kein großes Problem dar, insbesondere der Code konnte auch durch 'Pair-Programming' immer direkt validiert werden.
Das ständige gemeinsame Programmieren ist dabei überraschend effektiv, aber trotzdem nur bei kleinen zu schreibenden Codebeständen dauerhaft geeignet.
Für die Studienarbeit musste nicht viel Code geschrieben werden, weshalb das gemeinsame Programmieren sehr effektiv war.
\subsection{Arbeitsmittel}
Es wurde die IDE VisualStudio für den Python-Code verwendet, die weder besonders gut noch schlecht war.
Für die Latex-Ausarbeitung ist mit TeXstudio geschrieben, was die lokale und mit Git versionierte Bearbeitung der Dokumente erlaubt.
Das ist vor allem beim gleichzeitigen Schreiben sehr praktisch.
Allerdings hat Git bei Latex den Nachteil, das Merge-Konflikte nur sehr schwer zu beheben sind, weshalb Absprachen beim Schreiben getroffen werden müssen!

\section{Ausblick}
Es besteht die Möglichkeit, das GAN noch weiter zu optimieren.
So könnten wahrscheinlich noch bessere Ergebnisse erzeugt werden.
Außerdem ist es möglich mit besserer Hardware den gleichen Versuchsaufbau noch einmal durchzuführen und zu überprüfen, inwieweit bessere Ergebnisse erzielt werden können.

\section{Fazit}
Insbesondere das 'Pair Programming' war überraschend effektiv und kann bei Programmieraufgaben mit geringen Codeumfang definitiv weiterempfohlen werden.
Das Zeitmanagement des Projekts war auch gut, es ist auf alle Fälle sehr wichtig, früh anzufangen.
Auch ein Puffer von mindestens 4 Wochen kann in jedem Fall für ein Projekt dieser Länge empfohlen werden!