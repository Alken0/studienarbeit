%!TEX root = ../main.tex

\newcommand{\Titel}{Aufgabenstellung}
%!TEX root = ../main.tex

\begin{titlepage}
	\begin{center}
		\vspace*{-2.5cm}
		\hfill	\includegraphics[width=5cm]{images/dhbw.png}\\[5cm]
		
		{\Huge\scshape \Titel}\\[1cm]
		{\large für die Erstellung der Studienarbeit}\\[0.5cm]
		\vspace{1cm}
		
		\vfill
	\end{center}

	\begin{tabular}{l@{\hspace{2cm}}l}
		\bfseries Datum    & 17. Oktober 2021 \\
		\bfseries Betreuer & \Betreuer        \\
		\bfseries Kurs     & \Kursbezeichnung \\
		\bfseries Autoren  & \Autoren
	\end{tabular}

\end{titlepage}


\section{Titel der Arbeit}
Evaluierung des Lernerfolgs eines GANs mittels Generierung von geometrischen Figuren.

\section{Aufgabenstellung}
%Problembeschreibung
%Lösungsansatz
%geplante Vorgehensweise
\paragraph{Ausgangssituation} \hfill \\
Das Thema 'Neuronale Netzwerke' ist in den letzten Jahren immer populärer geworden.
Dabei handelt es sich um eine Art von 'Künstlicher Intelligenz', bei der verknüpfte virtuelle Neuronen komplexe Aufgaben wie Bilderkennung erledigen.

Leider werden für das Trainieren dieser Netze große Mengen an qualitativ hochwertigen Daten benötigt.
Diese Daten sind jedoch nicht immer verfügbar oder nur sehr aufwändig zu beschaffen.
Ein klassisches Beispiel für diese Datenknappheit ist der Medizinsektor, denn hier beschränkt das 'Patientengeheimnis' die freie Verbreitung von zum Beispiel Krankheitsanalysen.
Zudem sind in der Medizin bestimmte Krankheiten sehr selten und folglich wären benötigte Trainingsdaten für ein automatisches Analysetool nur in einer unzureichenden Anzahl vorhanden.

GANs erlauben an dieser Stelle die Generierung von synthetischen aber realistischen Testdaten.
Jedoch ist die Konfiguration von solchen GANs sehr komplex, da schon kleinste Anpassungen enorme Auswirkungen auf die Qualität des Ergebnisses haben können.
In der Studienarbeit werden die Konfigurationen deshalb weiter erforscht.

\paragraph{Aufgabenstellung} \hfill \\
Ziel der Studienarbeit ist die weitere Erforschung von möglichst optimalen Einstellungen für ein GAN, das geometrische Figuren produziert.
Bei den Figuren handelt es sich um Kreise, Dreiecke und Rechtecke in schwarz-weiß Bildern.
Die Figuren unterscheiden sich von Bild zu Bild in Größe oder Position, aber müssen je nach Bild-Label die richtige Form darstellen.
Die Eigenschaften der Figuren werden zufällig bestimmt, wobei Randkriterien betrachtet werden müssen.
So darf die Größe nicht so extrem gewählt sein, dass die Form nicht mehr erkennbar ist und die Figur muss eine Position besitzen, an der sie noch vollständig im Bild abgebildet ist.

Die Trainingsdaten für das GAN werden dafür extra generiert.
Dadurch entfällt ein aufwändiges labeln der einzelnen Formen und es bleiben mehr Freiheiten bei der genauen Definierung der Testdaten, falls diese im Nachhinein angepasst werden sollten (zum Beispiel zusätzliche Formen).

\paragraph{Geplante Vorgehensweise} \hfill \\
Die Studienarbeit soll grob in folgende Phasen untergliedert werden:

\begin{description}[style=nextline]
	\item[1. Recherche] 
	Literaturrecherche für ein grundsätzliches Verständnis von GANs und deren Funktionsweise
	
	\item[2. Entwicklung eines Prototypen-Generators für Trainingsdaten]
	Der Generator wird benötigt, um im nächsten Schritt das GAN darauf trainieren zu können.
	
	\item[3. Entwicklung eines Prototypen-GANs]
	Mit diesem Schritt ist der gesamte Trainingsablauf vorhanden und es können verschiedene Einstellungen versucht werden.
	
	\item[4. Weiterentwicklung von Generator und GAN]
	Hinzufügen von mehr Figuren beim Generator und daraus resultierenden Anpassungen im GAN.
	
	\item[Studienarbeit verfassen]
	Die Studienarbeit wird während der ganzen Zeit fortlaufend weitergeschrieben und angepasst.
\end{description} 
