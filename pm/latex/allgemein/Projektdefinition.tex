%!TEX root = ../main.tex

\newcommand{\Titel}{Projektdefinition}
%!TEX root = ../main.tex

\begin{titlepage}
	\begin{center}
		\vspace*{-2.5cm}
		\hfill	\includegraphics[width=5cm]{images/dhbw.png}\\[5cm]
		
		{\Huge\scshape \Titel}\\[1cm]
		{\large für die Erstellung der Studienarbeit}\\[0.5cm]
		\vspace{1cm}
		
		\vfill
	\end{center}

	\begin{tabular}{l@{\hspace{2cm}}l}
		\bfseries Datum    & 17. Oktober 2021 \\
		\bfseries Betreuer & \Betreuer        \\
		\bfseries Kurs     & \Kursbezeichnung \\
		\bfseries Autoren  & \Autoren
	\end{tabular}

\end{titlepage}


\section{Projektziel}
Ziel der Studienarbeit ist die Erforschung möglichst optimaler Konfigurationen eines GANs.
Das GAN soll scharz-weiß Bilder generieren, auf denen verschiedene und unterschiedliche geometrische Formen sichtbar sind.

\section{Projekttermin und Bearbeitungsdauer}
Die Studienarbeit ist bis zum \textbf{16. Mai 2022}.
Damit beträgt die theoretische Bearbeitungsdauer 7 Monate, wobei wahrscheinlich 2 Monate aufgrund der Praxisphase entfallen werden.

\section{Funktionen inklusive Alternativen und Bewertung}
Der Fokus der Studienarbeit ist nicht auf Implementierungen von GANs gerichtet, dafür werden Bibliotheken genutzt werden.
Stattdessen soll ausschließlich die Konfiguration der GANs erforscht werden.


\begin{description}[style=nextline]
	\item[Generierung von synthetischen Trainingsbildern]
	Damit die Konfiguration ausreichend untersucht werden kann, wird eine Vielzahl an gelabelten Trainingsbildern benötigt werden.
	Die Generierung erlaubt auch das 'Ein-/Ausschalten' bestimmter Eigenschaften in den Bildern, um eventuell andere Konfigurationen des GANs auszutesten.
	
	\textit{Alternative:}
	Es ist möglich, die Bilder manuell zu erstellen.
	Dafür können neue Bilder erzeugt werden und mit Bildbearbeitungstools verändert werden.
	
	\textit{Bewertung:}
	Die Generierung von Bildern ist kein neues Problem, weshalb Bibliotheken zur Generierung existieren werden.
	Ansonsten ist selbst eine komplette 'Eigenimplementierung' wahrscheinlich mit weniger Aufwand verbunden, als die Bilder manuell zu erstellen.
	
	\item[Generierung von Figuren nach Label]
	Jeder Figur wird ein Label zugeordnet, das die jeweilige Form beschreibt (ist ein Rechteck auf dem Bild zu sehen, wäre das Label 'Rechteck').
	Das GAN bekommt das Label als Eingabe für die Generierung und soll dann die entsprechende Form generieren.
	
	\textit{Alternative:}
	Die Generierung nach Label ist nicht unbedingt für die Studienarbeit notwendig.
	Folglich kann die Funktion im Notfall weggelassen werden.
	
	\textit{Bewertung:}
	Das Entfallen der Funktion ist zwar eine mögliche Option, aber sollte vermieden werden.
	
	\item[Generierung von Figuren mit unterschiedlicher Größe]
	Zwar sollen die Bilder gleich groß sein, die Formen in ihnen allerdings nicht.
	Das führt zu mehr Varietät zwischen den Bildern, wodurch das GAN mehr abstrahieren muss.
	Bei der Größenwahl ist zu beachten, dass die Form erkennbar bleibt.
	Das bedeutet, sie darf nicht zu groß oder zu klein sein.
	
	\textit{Alternative:}
	Die Generierung von Figuren unterschiedlicher Größe ist nicht unbedingt für die Studienarbeit notwendig.
	Folglich kann die Funktion im Notfall weggelassen werden.
	
	\textit{Bewertung:}
	Das Entfallen der Funktion ist zwar eine mögliche Option, aber sollte vermieden werden.
	
	\item[Generierung von Figuren mit unterschiedlicher Postition]
	Auch unterschiedliche Positionen zwingen das GAN die Figuren stärker zu abstrahieren.
	Bei der unterschiedlichen Positionen sind allerdings Randbedingungen zu beachten.
	So sollten die Figuren immer vollständig innerhalb des Bildes abgebildet sein.
	Ansonsten könnten zum Beispiel komplett leere Bilder entstehen.
	
	\textit{Alternative:}
	Die Generierung von Figuren mit unterschiedlicher Position ist nicht unbedingt für die Studienarbeit notwendig.
	Folglich kann die Funktion im Notfall weggelassen werden.
	
	\textit{Bewertung:}
	Das Entfallen der Funktion ist zwar eine mögliche Option, aber sollte vermieden werden.
	
	\item[Generierung von Figuren mit unterschiedlicher Rotation]
	Bei der Wahl der Rotation handelt es sich wieder um eine Eigenschaft, die es dem GAN schwerer machen sollte, korrekte Bilder zu generieren.
	Auch bei der Rotation ist zu beachten, dass die Figuren vollständig im Bild verbleiben.
	
	\textit{Alternative:}
	Die Generierung mit Rotation ist nicht unbedingt für die Studienarbeit notwendig.
	Folglich kann die Funktion im Notfall weggelassen werden.
	
	\textit{Bewertung:}
	Das Entfallen der Funktion ist zwar eine mögliche Option, aber sollte vermieden werden.
\end{description}
