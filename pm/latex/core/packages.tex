%!TEX root = ../main.tex

%%%%%%%%%%%%%%%%%%%%%%%%%%%%%%%%%%%%%%%%%%%%%%%%%%%%%%%%%%%%%%%%%%%%%%%%%%%
%% eigene Packages/Einstellungen
%%%%%%%%%%%%%%%%%%%%%%%%%%%%%%%%%%%%%%%%%%%%%%%%%%%%%%%%%%%%%%%%%%%%%%%%%%%

%Zeilenumbruch und mehr
\usepackage[activate]{microtype}

% Zeichencodierung
\usepackage[utf8]{inputenc}
\usepackage[T1]{fontenc}
\usepackage{verbatim}

% Zeilenabstand
\usepackage[onehalfspacing]{setspace}

% Index-Erstellung
\usepackage{makeidx}

% Lokalisierung (neue deutsche Rechtschreibung)
\usepackage[ngerman]{babel}

% Anführungszeichen 
\usepackage[babel,german=quotes]{csquotes}


% % Spezielle Tabellenform fuer Deckblatt
% \usepackage{longtable}
% \setlength{\tabcolsep}{10pt} %Abstand zwischen Spalten
% \renewcommand{\arraystretch}{1.5} %Zeilenabstand

% Grafiken
\usepackage{graphicx}

% Mathematische Textsaetze
\usepackage{amsmath}
\usepackage{amssymb}

% Pakete um Textteile drehen zu können, oder eine Seite Querformat anzeigen kann.
%\usepackage{rotating}
%\usepackage{lscape}



% Literaturverweise nach Harvard (mit deutschem und)
\usepackage[dcucite]{harvard}
\renewcommand{\harvardand}{und}

% Verschiedene Schriftarten
%\usepackage{goudysans}
%\usepackage{lmodern}
%\usepackage{libertine}
% \usepackage{palatino} 

% Hurenkinder und Schusterjungen verhindern
% http://projekte.dante.de/DanteFAQ/Silbentrennung
\clubpenalty=10000
\widowpenalty=10000
\displaywidowpenalty=10000


% Fussnoten
\usepackage[marginal, multiple, bottom, stable]{footmisc}



%Gleitumgebungen (Bilder, Tabellen, usw\ldots) lassen sich mit H an genau der
% definierten Stelle platzieren
\usepackage{float}

% für die vertikale Platzierung von Text in Tabellen
\usepackage{array}

% für die Darstellung des Euro-Symbols
\usepackage[right]{eurosym}

% für textumflossene Grafiken
\usepackage{wrapfig}

% eine Kommentarumgebung "k" (Handhabe mit \begin{k}<Kommentartext>\end{k},
% Kommentare werden rot gedruckt). Wird \% vor excludecomment{k} entfernt,
% werden keine Kommentare mehr gedruckt.
\usepackage{comment}
\specialcomment{k}{\begingroup\color{red}}{\endgroup}
%\excludecomment{k}

%%%%%%%%%%%%%%%%%%%%%%%%%%%%%%%%%%%%%%%%%%%%%%%%%%%%%%%%%%%%%%%%%%%%%%%%%%%
%% needed packages
%%%%%%%%%%%%%%%%%%%%%%%%%%%%%%%%%%%%%%%%%%%%%%%%%%%%%%%%%%%%%%%%%%%%%%%%%%%
\usepackage{babel}      % Sprachanpassungen für generierte Texte wie "Inhaltsverzeichnis" etc
\usepackage[T1]{fontenc}% Interne LaTeX Codierungen
\usepackage[dvipsnames,table]{xcolor} % Extending L A TEX’s color facilities
% \usepackage[babel, german=guillemets]{csquotes}   % Context sensitive quotation facilities.
% \usepackage{xspace}     % http://www.ctan.org/tex-archive/help/Catalogue/entries/xspace.html
\usepackage{array}      % http://www.ctan.org/tex-archive/help/Catalogue/entries/array.html
\usepackage{tabularx}   % http://www.ctan.org/tex-archive/help/Catalogue/entries/tabularx.html
\usepackage{eurosym}    % \euro
\usepackage{pdfpages}   % http://www.ctan.org/tex-archive/help/Catalogue/entries/pdfpages.html
\usepackage{needspace}  % http://www.tex.ac.uk/cgi-bin/texfaq2html?label=nopagebrk
\usepackage[onehalfspacing]{setspace}
\usepackage[bookmarksopen,bookmarksnumbered]{hyperref}
\usepackage{bookmark}   % Bookmarks for hyperref
\usepackage{graphicx}
\usepackage{hyperref}

%%%%%%COLORS%%%%%%%%%%%%%
\definecolor{simusorange}{HTML}{FF8000}
\definecolor{simusdark}{HTML}{F7893F}
\definecolor{darkblue}{HTML}{3B4FD1}
\definecolor{grasGreen}{RGB}{204, 255, 153}
\definecolor{lightBlue}{RGB}{153,204,255}
\definecolor{lightPurple}{RGB}{153, 153, 255}
\definecolor{codegreen}{rgb}{0,0.6,0}
\definecolor{codegray}{rgb}{0.5,0.5,0.5}
\definecolor{codepurple}{rgb}{0.58,0,0.82}
\definecolor{backcolour}{rgb}{0.95,0.95,0.92}
\definecolor{simusgreen}{HTML}{0EA458}
\definecolor{literaturrecherche}{HTML}{2F75B5}
\definecolor{testen}{HTML}{548235}
\definecolor{entwicklung}{HTML}{993366}
\definecolor{schriftlicheAusarbeitung}{HTML}{CB5F05}

%%%%%%%%%
% Glossar
\usepackage[
nonumberlist, %keine Seitenzahlen anzeigen
%acronym,      %ein Abkürzungsverzeichnis erstellen
%section,     %im Inhaltsverzeichnis auf section-Ebene erscheinen
toc,          %Einträge im Inhaltsverzeichnis
]{glossaries}

%Akronyme
\usepackage[printonlyused,footnote, withpage]{acronym} %maybe without page
%%%%%%%%%%%%%%

\usepackage{caption}
\usepackage{booktabs}
\usepackage{subfiles}
%\usepackage{subfigure}
%\usepackage{subfig}
\usepackage{subcaption} 
%\usepackage[list=true, font=large, labelfont=bf, labelformat=brace, position=top]{subcaption}
\usepackage{hyperref}
\usepackage[ngerman]{cleveref}

%%%%%%%%%%%%%%%%%%%%%%%%%%%%%%%%%%%%
%%%%%%%%%%%PseudoCode%%%%%%%%%%%%%%%
\usepackage{algorithm2e} %for psuedo code
\usepackage[lmargin=3.81cm,tmargin=2.54cm,rmargin=2.54cm,bmargin=2.52cm]{geometry}
%%%%%%%%%%%%%%%%%%%%%%%%%%%%%%%%%%%%
%%%%%%%%%%% Programmlistings setzen %%%%%%%%%%%%%%%
\usepackage{listings}   % http://www.ctan.org/tex-archive/macros/latex/contrib/listings/

% Wie sollen die Überschriften benannt werden:
\renewcommand{\lstlistingname}{Alg}

% Wie die Liste der Listings, s. \lstlistoflistings in bericht.tex
\renewcommand{\lstlistlistingname}{Liste der Algorithmen}

% So kann man einen Stil für alle  Algorithmen definieren
\lstdefinestyle{algoBericht}{
	numbers=left,              % Zeilennummern einfügen
	numberstyle=\tiny,         % wie werden sie gesetzt
	numbersep=5pt,             % Abstand der Nummern zum Text
	numberblanklines=false,    % bei Leerzeilen keine Nummer ausgeben (aber zählen)
	basicstyle=\sffamily\small,         % Wie soll der Algorithmus gesetzt werden
}

\lstdefinestyle{mystyle}{
	backgroundcolor=\color{white},   
	commentstyle=\color{codegreen},
	keywordstyle=\color{simusorange}\bfseries,
	numberstyle=\tiny\color{codegray},
	stringstyle=\color{codepurple},
	basicstyle=\ttfamily\footnotesize,
	breakatwhitespace=false,         
	breaklines=true,                 
	captionpos=b,                    
	keepspaces=true,                 
	numbers=left,                    
	numbersep=5pt,                  
	showspaces=false,                
	showstringspaces=false,
	showtabs=false,                  
	tabsize=2
}


%%%%%%%%%%%%%%%%%%%%%%%%%%%%%%%%%%%%%%%%%%%%%%%%%%%%%%%%%%%%%

\usepackage[headings]{fullpage}
\usepackage{url}
\usepackage{microtype}  % http://tug.ctan.org/tex-archive/macros/latex/contrib/microtype/
%\usepackage{lmodern}    % Computern-Modern Schriftfamilie
\usepackage{helvet}
\usepackage{amssymb}    % Symbole
\usepackage{framed}     % Framed or shaded regions that can break across pages.
% http://dante.ctan.org/tex-archive/help/Catalogue/entries/framed.html
% Benutzung siehe erklaerung.tex

\usepackage{makeidx}    % Erstellung eines Indexes
\makeindex

\usepackage{fancyhdr}  
%\pagestyle{fancy}
\fancyhead[L]{ }
\fancyhead[R]{ }
\fancyfoot[L]{}

\usepackage{wrapfig}    % Bilder von text umfliessen lassen

\usepackage[colorinlistoftodos]{todonotes}
% Einfache Verwaltung und Erstellung von TODO's Markierungen
% http://tug.ctan.org/tex-archive/macros/latex/contrib/todonotes/
% wichtige Paket-Optionen: disable




