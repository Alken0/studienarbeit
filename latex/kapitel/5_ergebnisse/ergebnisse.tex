% !TEX root = ../../main.tex
% !TeX spellcheck = de_DE

\chapter{Ergebnisse}

\section{Dense-GAN}
Über das Tensorboard können die generierten Daten in verschiedenen Ansichten angezeigt werden.
Die wichtigste ist die Hyperparameter-Ansicht.
Wie in Abbildung \ref{ergebnis:densegan-hyper} zu erkennen ist, werden nicht nur alle Parameterkombinationen angezeigt.
Es wird auch dargestellt, zu welchen Metriken diese Kombinationen geführt haben.
Damit lässt sich leicht evaluieren, welche Hyperparameter die besten Ergebnisse erzeugt haben.

\begin{figure}[H]
	\centering
	\includegraphics[width=0.75\textheight]{kapitel/5_ergebnisse/densegan/hyperparameter.PNG}
	\caption{Hyperparameter-Ansicht des Dense-GAN}
	\label{ergebnis:densegan-hyper}
\end{figure}

Auf der linken Seite sind alle Hyperparameter aufgelistet.
Die Abkürzung \textit{GEN} steht dabei für den Generator und \textit{DIS} repräsentiert den Diskriminator.
Außerdem können die einzelnen Graphen in einem Farbverlauf von Rot nach Blau einsortiert werden.
Dadurch kann der Einfluss von einzelnen Hyperparametern deutlich visualisiert werden.
\newline

In der Abbildung \ref{ergebnis:densegan-hyper} wurde nach der BatchSize sortiert.
Es ist deutlich zu erkennen, dass ein Wert von 32 (hier blau) bessere FID-Werte als 64 (hier rot) erzielt.
Im Gegensatz dazu steht die Smoothness und der Dropout.
Bei den gewählten Werten lassen sich kaum Tendenzen erkennen, wie in der Abbildung \ref{ergebnis:densegan-hyper-smoot-drop} visualisiert ist.

\begin{figure}[H]
	\centering
	\includegraphics[width=0.6\textheight]{kapitel/5_ergebnisse/densegan/hyperparameter_dropout_smooth.PNG}
	\caption{Einfluss von Smoothness und Dropout auf das Dense-GAN}
	\label{ergebnis:densegan-hyper-smoot-drop}
\end{figure}

\todo{vltt noch learning rate?}

\subsection{Bilder}
Wie bereits beschrieben, werden neben den Hyperparameter und Metriken auch generierte Bilddaten protokolliert.
Neben den FID-Werten, kann so die generierten Ergebnisse subjektiv Bewerter werden.
Die subjektiv besten Figuren sind in der Abbildung \ref{ergebnis:densegan-best} (a) zu sehen.
Die Figuren mit den besten FID-Werten werden in der Abbildung  \ref{ergebnis:densegan-best} (b) gezeigt.
Es lässt sich leicht erkennen, das die subjektive Bewertung und FID-Werte nicht übereinstimmen.

\begin{figure}[H]
	\centering
	\subfloat[][]{\includegraphics[width=0.45\linewidth]{kapitel/5_ergebnisse/densegan/good_example.png}}
	\qquad
	\subfloat[][]{\includegraphics[width=0.45\linewidth]{kapitel/5_ergebnisse/densegan/best_fid.png}}
	\caption{Beste Ergebnisse bemessen nach subjektiver Bewertung und FID-Werte}
	\label{ergebnis:densegan-best}
\end{figure}

Da das erste Beispiel gute Ergebnisse erzeugt, wurde ein GAN mit den gleichen Hyperparametern für 1000 Epochen trainiert.
Auch wenn eine große Menge an Bildartefakte zu erkennen ist, stechen die Kanten der Formen deutlicher heraus (siehe Abbildung \ref{ergebnis:densegan-good-example-long}).

\begin{figure}[H]
	\centering
	\includegraphics[height=0.4\textheight]{kapitel/5_ergebnisse/densegan/good_example_long.png}
	\caption{Ergebnis von 1000 Epochen Training}
	\label{ergebnis:densegan-good-example-long}
\end{figure}

\section{DC-GAN-Late-Label}
\begin{itemize}
	\item Tensorboard Graph -> beste Hyperparameter?
	\item 'beste' Bilder
\end{itemize}

\section{DC Generator und Dense Discriminator}

\section{Dense Generator und DC Discriminator}

\section{DC-GAN-Medical-Inspired}
\begin{itemize}
	\item Tensorboard grafik
	\item Beste Bilder
	\item Hyperparameter Analyse
\end{itemize}
