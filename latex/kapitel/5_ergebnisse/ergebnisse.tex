% !TEX root = ../../main.tex
% !TeX spellcheck = de_DE

\chapter{Ergebnisse}

\section{DC-GAN-Late-Label}
Die Architektur dieses Modells basiert nicht auf anderen Papern oder Tutorials. \todo{Fabi zitieren? :D}
Stattdessen wird die Idee ausprobiert, die Convolutional Layer des Discriminators ausschließlich auf die Bilder anzuwenden.
Das Label wird dann erst im letzten Schritt mit dem Ergebnis der Convolutional Layer verknüpft und bewertet.
Die Entscheidung ist mit dem Verhalten der Convolutional Layern zu begründen.
Denn die Layer extrahieren Zusammenhänge von nebeneinanderliegenden Pixeln.
Diese Zusammenhänge sind jedoch nicht natürlich bei einem synthetischen Latent-Vektor gegeben.

\subsection{Architektur}
Die Besonderheit der Architektur liegt in der späten Fusionierung von Bild- und Labeldaten beim Discriminator.
Ein Diagramm mit der genauen Architektur befindet sich im Anhang (Discriminator \cref{architecture:dcgan-dis} und Generator \cref{architecture:dcgan-gen}).


Allgemein werden primär Convolutional und Convolutional Transposed Layer genutzt.
Beim Discriminator besteht eine Schicht aus einer Kombination von Conv2d, LeakyReLu (mit einem alpha von 0.2) und Dropout (beeinflusst vom Hyperparameter).
Der Generator besteht aus Schichten von Conv2DTranspose, BatchNormalisation und LeakyReLu (mit einem alpha von 0.2).

\subsection{Hyperparameter}
Die Hyperparameter wurden in einem Gridsearchverfahren, das heißt in allen Kombinationen angewendet.
\begin{table}[H]
	\centering
	\begin{tabular}{l l}
		Name                        & Werte            \\ \hline
		Epochen                     & 100              \\
		Batch Size                  & 16, 32           \\
		Dropout                     & 0.3, 0.4         \\
		Smoothness                  & 0, 0.1           \\
		Learning Rate Discriminator & 2e-3, 2e-4, 3e-4 \\
		Learning Rate Generator     & 2e-3, 2e-4, 3e-4
	\end{tabular}
	\caption{Hyperparameter für das Training des 'DC-GAN-Late-Label's}
\end{table}

\subsection{Ergebnisse}
\begin{itemize}
	\item Tensorboard Graph
	\item 'beste' Bilder
\end{itemize}

\subsection{Zusammenfassung}
\begin{itemize}
	\item ignoriert Label
	\item keine Figuren
	\item kein mode-collapse
\end{itemize}

\section{DC-GAN-COVID-Inspired}
\subsection{Allgemein}
Die Architektur dieses Modells orientiert sich an einem Paper zu COVID Fällen \cite{inspiration-dc-gan-2}.


\begin{itemize}
	\item deep convolutional
	\item woher idee \cite{inspiration-dc-gan-2}
	\item welche anpassungen
\end{itemize}

\subsection{Architektur}
\begin{itemize}
	\item Was ist besonders an Architektur?
	\item wirklich nochmal in Textform beschreiben?
	\item alpha von LeakyRelu wird nicht angezeigt
	\item Discriminator 
	\item Generator \cref{architecture:dcgan-gen}
\end{itemize}

\subsection{Hyperparameter}
\begin{table}[H]
	\centering
	\begin{tabular}{l l}
		Name                        & Werte            \\ \hline
		Epochen                     & 100              \\
		Batch Size                  & 16, 32           \\
		Dropout                     & 0.3, 0.4         \\
		Smoothness                  & 0, 0.1           \\
		Learning Rate Discriminator & 2e-3, 2e-4, 3e-4 \\
		Learning Rate Generator     & 2e-3, 2e-4, 3e-4
	\end{tabular}
	\caption{Hyperparameter für das Training des DC-GANs}
\end{table}

\subsection{Ergebnisse}
\begin{itemize}
	\item Tensorboard grafik
	\item Beste Bilder
	\item Hyperparameter Analyse
\end{itemize}

\section{DENSE-GAN}
\subsection{Architektur}
Sowohl die Architektur zum Generator \cref{architecture:densegan-gen} als auch zum Discriminator \cref{architecture:densegan-dis} ist in den Anlagen als Diagramm zu finden.


Die Architektur ist inspiriert von einem Tutorial für den MNIST Datenset \cite{inspiration-dense-gan}.
Da die Datensätze sehr ähnlich sind, kann die Struktur übernommen werden.


\subsection{Hyperparameter}
Für das Training wurde eine Anzahl an vordefinierten Hyperparameterkombinationen ausprobiert.

\begin{table}[H]
	\centering
	\begin{tabular}{l l}
		Name                        & Werte            \\ \hline
		Epochen                     & 100              \\
		Batch Size                  & 16, 32           \\
		Dropout                     & 0.3, 0.4         \\
		Smoothness                  & 0, 0.1           \\
		Learning Rate Discriminator & 2e-3, 2e-4, 3e-4 \\
		Learning Rate Generator     & 2e-3, 2e-4, 3e-4
	\end{tabular}
	\caption{Hyperparameter für das Training des Dense-GANs}
\end{table}



\subsection{Ergebnisse}

