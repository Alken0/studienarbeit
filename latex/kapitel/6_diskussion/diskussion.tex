% !TEX root = ../../main.tex
% !TeX spellcheck = de_DE

\chapter{Diskussion}
%TODO Aufbau nach: https://www.scribbr.de/aufbau-und-gliederung/diskussion-bachelorarbeit/

In Rahmen der Studienarbeit wurden diverse CGAN Architekturen zur Generierung von geometrischen Figuren untersucht.
Bei den Figuren handelt es sich um Kreise, Quadrate und Dreiecke, die in unterschiedlicher Größe auf dem Bild verteilt sein können.
Je nach Eingabelabel soll dann das GAN bestimmte Figuren generieren.
Für den Vergleich wurden alle Architekturen mit der jeweiligen Hyperparametersuche für 100 Epochen auf die Bilder trainiert.
In diesem Kapitel werden die Ergebnisse analysiert, interpretiert und eingeordnet.

\section{Zusammenfassung der Ergebnisse}
Die besten Ergebnisse erzielt das Dense GAN, das ausschließlich aus Dense Layern besteht.
Es ist in der Lage erkennbare Figuren zu erzeugen, die vom vorgegebenen Label abhängig sind.
Allerdings hat das Dense GAN Schwierigkeiten mit Rauschen.
So sind immer wieder kleinere Pixelfragmente im Hintergrund zu sehen, die nicht zur eigentlichen Figur gehören.
\newline

Alle anderen Architekturen besitzen Convolutional Layer, wobei es sich entweder Discriminator, Generator oder bei beiden Netze um Deep Convolutional Networks handelt.
All diese Architekturen haben deutlich weniger Probleme mit Hintergrundrauschen.
Jedoch ist die Qualität der Bilder trotzdem deutlich schlechter als beim Dense GAN.
Hauptproblem ist, dass immer ein bis zwei Einheitsfiguren erzeugt werden, statt den gewünschten Figuren.
Die Einheitsfiguren ähneln am ehesten dem Quadrat oder Kreis und werden vom GAN an verschiedenen Stellen in unterschiedlichen Größen im Bild platziert.
Zwar handelt es sich nicht um einen Mode Collapse, weil sich die Bilder unterscheiden, das Label wird aber auch nicht erkennbar in den Gernierungsprozess einbezogen.

\section{Interpretation der Ergebnisse}
Zunächst zeigt sich, dass die Kombination aus Dense und Deep Convolutional GAN keinen Vorteil gegenüber Architektur-Reinformen verschafft.
Stattdessen verhalten sie sich sehr ähnlich zu den Deep Convolutional GANs.

Die besten Ergebnisse erzielt das Dense GAN.
Alle anderen 
\subsubsection{Convolutional Layer im Vergleich zu Dense Layern}
Die Ergebnisse zeigen deutlich, dass die Convolutinonal Layer keinen positiven Einfluss auf das Endresultat haben, auch nicht in Kombination mit Dense Layern.
Stattdessen neigen sie viel mehr zu einer Einheitsfiguren, die eine Mischung aus Kreis und Quadrat darstellt.
Alleine das Dense-GAN schafft, Figuren korrekt nach Label darzustellen.

Hingegen ein positiver Aspekt bei der Vewendung von Convolutional Layern ist die Absenz oder zumindest deutliche Reduktion von Hintergrundrauschen.
In diesem Bereich hat das Dense GAN deutlich mehr Probleme.

\subsubsection{Labelsensitivität}
Wie bereits angesprochen zeigen die Ergebnisse, dass alle GANs bis auf das Dense-GAN erhebliche Probleme haben, korrekte Bilder nach Label zu erzeugen.
Statt verschiedenen Formen wird oftmals nur eine Standardform produziert, die einer Mischung aus Kreis und Quadrat ähnlich ist.
Es handelt sich dabei zwar um keinen mode-collapse, da die Figuren immer unterschiedlich aussehen, aber die GANs sind nicht Label-sensitiv.

Das könnte daran liegen, dass sich die Figuren bei geringer verpixelter Auflösung ähneln.
Insbesondere, wenn die Figuren kleiner auf dem Bild abgebildet sind, als die maximale Bildgröße erlauben würde.
Dadurch könnte ein Ungleichgewicht gegenüber der Dreiecksfigur entstehen, was dann den Lernprozess beeinflusst.
\newline



Interessant ist auch die 

\begin{enumerate}
	\item GAN hat Label nicht richtig beachtet
	\begin{enumerate}
		\item Formen (Kreis, Quadrat) zu ähnlich?
	\end{enumerate}

	\item gemischte Architektur kein Vorteil
	\begin{enumerate}
		\item weil dc-gan allgemein nicht gut?
		\item weil sie nicht gut zusammen trainieren?
	\end{enumerate}

	\item FID
	\begin{enumerate}
		\item FID Wert nicht so aussagekräftig wie angepriesen
		\item weil Bilder untypisch klar definiert sind?
	\end{enumerate}
\end{enumerate}

\section{Beschränkung der Forschung}
\begin{enumerate}
	\item Hardware
	\begin{enumerate}
		\item geringe Anzahl an Hyperparametern
		\item eventuell zu kleine Netze?
	\end{enumerate}
\end{enumerate}

\section{Ausblick}
% TODO Empfehlungen für weiterführende Forschung
\begin{enumerate}
	\item Bilder mehr preprocessen?
	\begin{enumerate}
		\item oft werden Bilder gecroppt/skaliert um bestimmte Größe zu passen
		\item unterschiedliche Position der Formen zu schwer?
		\item Viereck und Kreis ähneln sich relativ stark im Vergleich zum Dreieck
		\item durch sehr klare definierte Formen sehr hohe Ansprüche ans Ergebnis
	\end{enumerate}

	\item Architektur mit erst Dense dann Großflächigen Convolutional Layer am Ende gegen Rauschen?

	\item bessere Hardware verwenden
	\begin{enumerate}
		\item Genetic Algorithm
		\item evt sogar Architektur allgemein die Layer als HP definieren
		\item mehr Epochen trainieren
	\end{enumerate}
\end{enumerate}