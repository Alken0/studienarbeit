%!TEX root = ../../main.tex

\chapter{Einleitung}

Das Thema 'Neuronale Netzwerke' ist in den letzten Jahren immer populärer geworden.
Dabei handelt es sich um eine Art von 'Künstlicher Intelligenz', bei der verknüpfte virtuelle Neuronen komplexe Aufgaben wie Bilderkennung erledigen.

Leider werden für das Trainieren dieser Netze große Mengen an qualitativ hochwertigen Daten benötigt.
Diese Daten sind jedoch nicht immer verfügbar oder nur sehr aufwändig zu beschaffen.
Ein klassisches Beispiel für diese Datenknappheit ist der Medizinsektor, denn hier beschränkt das 'Patientengeheimnis' die freie Verbreitung von zum Beispiel Krankheitsanalysen.
Zudem sind in der Medizin bestimmte Krankheiten sehr selten und folglich wären benötigte Trainingsdaten für ein automatisches Analysetool nur in einer unzureichenden Anzahl vorhanden.

GANs erlauben an dieser Stelle die Generierung von synthetischen aber realistischen Testdaten.
Jedoch ist die Konfiguration von solchen GANs sehr komplex, da schon kleinste Anpassungen enorme Auswirkungen auf die Qualität des Ergebnisses haben können.
In der Studienarbeit werden die Konfigurationen deshalb weiter erforscht.
\newline

Ziel der Studienarbeit ist die weitere Erforschung von möglichst optimalen Einstellungen für ein GAN, das geometrische Figuren produziert.
Bei den Figuren handelt es sich um Kreise, Dreiecke und Rechtecke in schwarz-weiß Bildern.
Die Figuren unterscheiden sich von Bild zu Bild in Größe oder Position, aber müssen je nach Bild-Label die richtige Form darstellen.
Die Eigenschaften der Figuren werden zufällig bestimmt, wobei Randkriterien betrachtet werden müssen.
So darf beispielsweise die Größe nicht so extrem gewählt sein, dass die Form nicht mehr erkennbar ist und die Figur muss eine Position besitzen, an der sie noch vollständig im Bild abgebildet ist.
