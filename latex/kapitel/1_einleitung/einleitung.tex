% !TEX root = ../../main.tex
% !TeX spellcheck = de_DE

\chapter{Einleitung}
\section{Ausgangssituation}
Das Thema 'künstliche neuronale Netze' ist über die letzten Jahre immer populärer geworden.
Die Netze sind in der Lage diverse Aufgaben zu erledigen, von Gesichts- oder Betrugserkennung, Übersetzungsaufgaben oder die Generierung von täuschend echten Daten.
\newline

Insbesondere die Datengenerierung mithilfe von \acrfull{GAN} ist interessant, da täuschend echte Daten erzeugt werden können, ohne dass Menschen im Trainingsprozess ständig für die Bewertung eingreifen müssen.
Die Daten können dann unter anderem für das Training anderer künstlicher neuronaler Netze verwendet werden, die Auflösung von Bildern verbessert oder auch neuartige Bilder generiert werden.

\section{Problemstellung}
Allerdings ist es nicht trivial mithilfe von künstliche neuronale Netze gute Ergebnisse in den genannten Bereichen zu erzielen.
Denn statt vordefinierte Schritte zu programmieren, werden die Netze auf eine große Datenmenge trainiert.
Das Training ist neben den Daten auch stark von zusätzlichen Konfigurationen, sogenannten Hyperparametern abhängig.
Diese Vielzahl an Einflussfaktoren macht die Definition einer guten Trainingsumgebung sehr schwer.
\newline

Eine Datengenerierung kann außerdem durch Bedingungen zur Beeinflussung der Generierungen noch komplexer werden.
So soll dann zum Beispiel bei einer Datengenerierung nicht nur irgendein authentisches Bild generiert werden, sondern ein Bild mit einem schwarzen Kreis.
\newline

Die Lösung zu diesem Problem besteht oftmals in der Wiederverwendung bereits bekannter Netze inklusive deren Hyperparametern.
Das funktioniert allerdings nur solange es Überschneidungen bei den verwendeten Daten und Anforderungen gibt.

\section{Zielsetzung}
Für die Studienarbeit wird ein neues Datenset eingeführt, Bilder geometrischer Figuren enthält, das es in der Form nicht gibt.
Bei den Figuren handelt es sich um Dreiecke, Quadrate und Kreise, die sich in Größe und Position unterscheiden.
Es soll dann ein möglichst gutes Netz inklusive Hyperparametern gefunden werden, das eine beeinflussbare Generierung von ähnlichen Bildern erlaubt.
Beeinflussbar soll die abgebildete Figur sein, nicht jedoch die Position oder Größe.

Ziel ist es, eine Grundlage für zukünftige Forscher oder Entwickler zu schaffen, die ähnliche Probleme lösen sollen.
Die Ergebnisse der Arbeit dienen dann als Basis für eine schnelle und fundierte Auswahl von verschiedenen Netzarchitekturen und zugehörigen Hyperparametern.

\section{Vorgehensweise}
Im Rahmen der Arbeit werden zunächst die Trainingsdaten generiert.
Danach erfolgt die Festlegung von Bewertungskriterien an die erzeugten Bilder.
Hierbei stützen wir uns als objektive Metrik vor allem auf den FID-Wert.
Bei den Architekturen handelt es sich um ein Dense GAN und ein Deep Convolutional GAN, die dann im Trainingsprozess inklusive Hyperparametersuche miteinander verglichen werden.
Die Trainingsergebnisse werden dann im Anschluss analysiert und Besonderheiten 
